\section{Introducción}
  
  Uno de los conceptos centrales para la computación, tal y como la conocemos hoy en día, es el de \emph{sistema operativo}. Se trata de la pieza de software que actúa como interfaz entre las aplicaciones de usuario, concebidas para realizar las más diversas tareas, y el hardware que efectúa las operaciones necesarias para llevarlas a cabo. Gracias a su existencia, un único procesador es capaz de ejecutar varias tareas de manera concurrente, lo cual se conoce como \emph{multitasking}. Este paradigma exige administrar cuidadosamente los recursos compartidos de hardware, brindar servicios que permitan la interacción entre las tareas y el sistema, y prestar especial atención a la seguridad, asegurándose de que ningún proceso sea capaz de alterar indebidamente el estado de otro.
  
  En el transcurso de este trabajo, nos proponemos construir un sencillo sistema operativo de 32 bits para la arquitectura \emph{x86} de Intel, que sea capaz de cumplir las funciones recién enunciadas para un número limitado de tareas concurrentes, empleando los mecanismos que brinda el \emph{modo protegido} de dicha arquitectura. Para llevarlo a cabo, utilizaremos como entorno de pruebas el software \emph{Bochs}, que nos permite emular una PC con un procesador x86 completamente funcional. Nuestro objetivo es explorar y conocer los pormenores del funcionamiento interno del procesador a nivel de arquitectura, aplicando los conceptos sobre \emph{System Programming} adquiridos en clase, y comprender así lo que una computadora hace “tras bambalinas” cada vez que la utilizamos.
