\section{El sistema}
  
  El sistema operativo que desarrollaremos será la implementación de un juego de dos jugadores. Cada uno de ellos podrá lanzar \emph{perros} (hasta un máximo de 8 simultáneamente), que serán representados en el sistema mediante tareas. Estos perros se moverán por un \emph{mapa}, que será un sector de la memoria, en busca de \emph{huesos}, partiendo desde una posición inicial, la \emph{cucha}, que estará fija para cada jugador. En algún momento, los perros deberán volver a la cucha y depositar los huesos recolectados, tras lo cual se echarán a dormir, siendo eliminados del sistema.
  
  En primer lugar, el sistema tendrá que disponer las estructuras necesarias para administrar correctamente la memoria, utilizando para este fin los mecanismos de \emph{paginación} y \emph{segmentación} brindados por el procesador. Además, los jugadores podrán realizar acciones, como desplazarse por el mapa, lanzar nuevos perros y darles órdenes; para esto utilizarán el teclado, interrumpiendo al sistema, que deberá atenderlos. El sistema también brindará servicios a los perros en ejecución, con lo cual les permitirá moverse, olfatear en busca de huesos, cavar para obtenerlos, y recibir órdenes de sus amos. Por otra parte, será necesario implementar mecanismos para repartir el tiempo de ejecución entre los perros (es decir, un \emph{scheduler}) y para retirar del sistema a aquellos que realicen acciones indebidas y generen excepciones.
